\section{Conclusiones}

De acuerdo con los resultados obtenidos en las encuestas y la síntesis general de la  \autoref{sintesis}, se puede concluir que una estrategia efectiva para mejorar la comprensión oral del inglés en la población bilingüe de Colombia es la integración de herramientas tecnológicas innovadoras que combinen metodologías interactivas y motivadoras con el uso de inteligencia artificial.
\\
\\
Los resultados de las encuestas revelan que los participantes consideran que el desconocimiento del inglés ha obstaculizado su rendimiento académico y sus oportunidades personales. Además, se reconoce el potencial del Procesamiento de Lenguaje Natural (PLN) como catalizador en el proceso de enseñanza-aprendizaje, destacando su capacidad para personalizar la experiencia y mejorar la calidad educativa.
\\
\\
En este contexto, el prototipo web English with Ogloc demostró ser funcional y atractivo para los usuarios. La facilidad de acceso a las lecciones, la claridad de los contenidos generados, y el uso del micrófono como herramienta para practicar la producción oral, son indicios de que un aplicativo web con enfoque en comprensión y producción oral puede ser una solución efectiva. Además, se presentó en el IV Encuentro Regional de Tesistas, donde también tuvo una aceptación positiva por parte del público y de los evaluadores. Adicionalmente, la incorporación de elementos de gamificación contribuyeron a elevar los niveles de motivación y compromiso.
\\
\\
Por tanto, para dar respuesta a la problemática de cómo mejorar la comprensión oral del inglés en Colombia, se recomienda fortalecer el desarrollo e implementación de plataformas que integren tecnologías emergentes, diseños centrados en el usuario y metodologías activas. Esta aproximación permite no solo facilitar la exposición al inglés hablado en contextos diversos, sino también impulsar una práctica constante que refleje el esfuerzo del usuario mediante retroalimentación y recompensas simbólicas. Todo ello apunta a cerrar las brechas existentes y potenciar las competencias comunicativas en inglés dentro de la población bilingüe colombiana.

\newpage
\section{Trabajos futuros}

Durante el desarrollo del prototipo se lograron identificar diferentes aspectos que complementarían y ayudarían al prototipo web a tener una mejora significativa:

\begin{itemize}
    \item Implementar un servicio para la funcionalidad de speech-to-text del prototipo. utilizar la librería gratuita Speech Web API lleva consigo a limitar el uso del prototipo a solo 2 navegadores en dispositivos de escritorio, lo cual al utilizar un servicio en la nube (de paga) esta limitación desaparecería, ademas que la calidad de la transcripción tendría una mejora significativa.

    \item Realizar el despliegue del prototipo en un servidor con GPU y una mayor cantidad de memoria RAM, para obtener un menor tiempo de generación de las lecciones, pues los modelos al ser procesados mediante GPU procesan a una mayor velocidad, y al tener mas memoria disponible, mejoraría la concurrencia en Celery.

    \item Realizar un ajuste fino a los modelos implementados con el objetivo de mejorar la generación, y la diversidad de las preguntas, llegando a poder realizar distintos tipos de preguntas o diversificando aun mas el contenido de las lecciones.
    
\end{itemize}