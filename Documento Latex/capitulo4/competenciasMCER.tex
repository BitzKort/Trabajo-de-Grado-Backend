\subsection{Competencias MCER}

El Marco Común Europeo de Referencia (MCER) es un estándar internacional que proporciona una base común para diseñar programas de lenguas, exámenes y materiales didácticos. Este marco define lo que los estudiantes deben aprender para comunicarse eficazmente en una lengua extranjera, incluyendo las competencias en las distintas destrezas lingüísticas, así como los conocimientos del idioma y la comprensión cultural. El MCER establece seis niveles de dominio (A1, A2, B1, B2, C1 y C2) que permiten medir el progreso del estudiante de forma objetiva.
\\
\\
Dado que el alcance de este proyecto está dirigido a personas en Colombia que deseen mejorar sus conocimientos del idioma inglés y que poseen un nivel de competencia entre A2 y B1, es importante especificar los descriptores de estos niveles. Para ello, tomamos como referencia la guía oficial del Consejo de Europa y el Instituto Cervantes \cite{mcer2002}.

\begin{table}[H]
\centering
\small
\begin{tabular}{|p{4cm}|p{10cm}|}
\hline
\textbf{Nivel} & \textbf{Descripción  escala global} \\
\hline
\textbf{A2 (Usuario básico)} & Es capaz de comprender frases y expresiones de uso frecuente relacionadas con áreas de experiencia que le son especialmente relevantes (información básica sobre sí mismo y su familia, compras, lugares de interés, ocupaciones, etc.). Sabe comunicarse a la hora de llevar a cabo tareas simples y cotidianas que no requieran más que intercambios sencillos y directos de información sobre cuestiones que le son conocidas o habituales. Sabe describir en términos sencillos aspectos de su pasado y su entorno, así como cuestiones relacionadas con sus necesidades inmediatas. \\
\hline
\textbf{B1 (Usuario independiente)} & Es capaz de comprender los puntos principales de textos claros y en lengua estándar si tratan sobre cuestiones que le son conocidas, ya sea en situaciones de trabajo, de estudio o de ocio. Sabe desenvolverse en la mayor parte de las situaciones que pueden surgir durante un viaje por zonas donde se utiliza la lengua. Es capaz de producir textos sencillos y coherentes sobre temas que le son familiares o en los que tiene un interés personal. Puede describir experiencias, acontecimientos, deseos y aspiraciones, así como justificar brevemente sus opiniones o explicar sus planes. \\
\hline
\end{tabular}
\caption{Descriptores generales de los niveles A2 y B1 del MCER}
Fuente: Consejo de Europa MCER \cite{mcer2002}
\label{tab:descriptores-mcer}
\end{table}

\subsection{Temas de ingles}

\section*{Introducción}

En esta sección se presenta una recopilación de temas gramaticales específicos correspondientes a los niveles A2 (pre-intermedio) \cite{britisha1a2grammar} y B1 (intermedio) \cite{britishb1b2grammar} del MCER, basándose en los recursos proporcionados por el British Council. Estos temas son fundamentales para identificar y justificar el nivel de competencia.

\section*{Nivel A2 – Usuario Básico (Pre-intermedio)}

Según el British Council \cite{britisha1a2grammar}, los estudiantes en el nivel A2 deben ser capaces de comprender y utilizar estructuras gramaticales básicas en contextos cotidianos. Algunos de los temas que están en este nivel son:

\begin{itemize}[leftmargin=*, label=--]
    \item \textbf{Artículos:} uso de \textit{a}, \textit{an}, \textit{the} y casos donde no se utiliza artículo.
    \item \textbf{Adjetivos terminados en \textit{-ed} y \textit{-ing}:} diferencias entre \textit{bored} y \textit{boring}.
    \item \textbf{Adverbios de frecuencia:} \textit{always}, \textit{usually}, \textit{sometimes}, \textit{never}.
    \item \textbf{Comparativos y superlativos:} \textit{older}, \textit{the oldest}, \textit{more interesting}, \textit{the most interesting}.
    \item \textbf{Pronombres:} personales, posesivos, demostrativos y reflexivos.
    \item \textbf{Tiempos verbales:} presente simple, pasado simple, futuro con \textit{will} y \textit{going to}.
\end{itemize}

\section*{Nivel B1 – Usuario Independiente (Intermedio)}

En el nivel B1 \cite{britishb1b2grammar}los estudiantes deben ser capaces de comprender y utilizar estructuras gramaticales complejas en una variedad de contextos. Algunos de los temas que están en este nivel son:

\begin{itemize}[leftmargin=*, label=--]
    \item \textbf{Adjetivos gradables y no gradables:} uso de intensificadores como \textit{a bit}, \textit{really}, \textit{absolutely}.
    \item \textbf{Uso de mayúsculas y apóstrofes:} reglas de puntuación y ortografía.
    \item \textbf{Tiempos verbales:} presente perfecto, pasado perfecto, futuro continuo, pasado continuo.
    \item \textbf{Voz pasiva:} formación y uso en diferentes tiempos verbales.
    \item \textbf{Condicionales:} segundo y tercer condicional para situaciones hipotéticas.
    \item \textbf{Verbos frasales (phrasal verbs):} uso y significado en contextos variados.
\end{itemize}

\section*{Relevancia de los temas en el prototipo}

La identificación y comprensión de estos temas gramaticales son esenciales para evaluar y justificar el nivel de competencia lingüística en inglés. Utilizando los recursos proporcionados por el British Council \cite{britishcouncilgrammar}, se puede asegurar una base sólida y confiable para el aprendizaje y la enseñanza del idioma.

Si bien se han identificado y categorizado los temas gramaticales más relevantes de los niveles A2 y B1, es importante destacar que su incorporación en el prototipo no obedece a fines explicativos o pedagógicos aislados. Dado que el prototipo está diseñado para generar textos, preguntas y respuestas, la selección de estos contenidos tiene como propósito principal profundizar en el perfil de la población objetivo, es decir, a la población bilingüe Colombiana cuyo nivel de inglés se sitúa entre A2 y B1.

Además, los temas identificados sirven como base para justificar lingüísticamente la validez de los textos generados por el prototipo. Esta justificación se respalda mediante la encuesta de niveles de inglés (capítulo 7), la cual fue aplicada con el fin de evaluar la adecuación del contenido lingüístico incluido en las lecciones. Dicha encuesta fue dirigida exclusivamente a docentes con experiencia en la enseñanza del inglés, lo que garantiza un respaldo experto en la validación del nivel y calidad del contenido generado.

\subsection{Metodolog\'ias actuales}
La enseñanza del inglés como lengua extranjera ha evolucionado a lo largo del tiempo, adoptando diversos enfoques y metodologías que van desde modelos tradicionales hasta propuestas más centradas en la comunicación y en el estudiante. Entre las metodologías más empleadas se encuentran el método gramatical-traductivo, \cite{metodogramaticatrad}, basado en el análisis detallado de las reglas gramaticales y sus excepciones, aplicándose a través de la traducción de oraciones y textos; el enfoque comunicativo, \cite{enfoquecomunicativo}, que promueve el uso de textos, grabaciones y actividades que reproducen situaciones reales fuera del aula; el aprendizaje basado en tareas (TBLT), \cite{TBLT}, que se centra en la realización de tareas significativas en la lengua meta para fomentar la comunicación efectiva; y el aprendizaje blended (Blended Learning) \cite{salinas2010entrenamiento}, que integra la instrucción presencial con el uso de recursos digitales para enriquecer el proceso educativo.
\\
\\
Estas metodolog\'ias han contribuido significativamente al desarrollo de competencias ling\"u\'isticas; sin embargo, presentan ciertos desaf\'ios. Muchas veces, los enfoques tradicionales resultan mon\'otonos, poco motivadores y desconectados de los intereses y contextos reales de los estudiantes. Incluso las propuestas m\'as modernas pueden verse limitadas si no logran captar la atenci\'on del aprendiz o si no fomentan su participaci\'on activa y constante.
\\
\\
Ante estas limitaciones, ha emergido con fuerza la gamificaci\'on como una estrategia pedag\'ogica innovadora que transforma el aprendizaje del ingl\'es en una experiencia l\'udica, interactiva y motivadora. La gamificaci\'on se basa en incorporar elementos propios del juego (puntos, niveles, recompensas, retos, etc.) en contextos educativos, con el fin de aumentar el compromiso, la autonom\'ia y la motivaci\'on del estudiante.
\\
\\
En este punto también es necesario aclarar la diferencia entre aprender un lenguaje como lengua materna y aprender un lenguaje de forma extranjera (segundo idioma), pues implica procesos distintos \cite{flores2015using}. Mientras que la primera se adquiere de manera natural y espont\'anea en entornos ricos en lenguaje, desde una edad temprana y sin instrucci\'on formal, el aprendizaje de una lengua extrangera, gneralmente ocurre en contextos m\'as restringidos, como el aula, y requiere esfuerzos conscientes y concientes. Esta diferencia impacta directamente en las metodologías empleadas, y resalta la necesidad de estrategias más efectivas, como el blend learning más gamificación, que puedan simular ambientes inmersivos y motivadores similares a los del aprendizaje de la lengua materna.
\\
\\
Considerando los desafíos en la enseñanza del inglés como lengua extranjera, se utilizo para el proyecto el aprendizaje blended con técnicas de gamificación. Esta combinación permite integrar la flexibilidad de los recursos digitales con la interacción presencial, al mismo tiempo que aprovecha el potencial motivador de los elementos lúdicos para fomentar un aprendizaje significativo, contextualizado y motivando al estudiante. Como se observa en la \autoref{tab:descriptores-mcer}, las personas que se encuentran en los niveles A2 y B1 del MCER presentan necesidades específicas en cuanto al desarrollo de sus competencias comunicativas. Por ello, dichas características metodológicas resultan especialmente valiosas para fortalecer las competencias en el contexto colombiano, proporcionando un entorno educativo que favorece el avance progresivo en el dominio del idioma.
