La gamificación ofrece una amplia gama de técnicas que pueden ser aplicadas en contextos educativos para fomentar la motivación, el compromiso y el aprendizaje significativo. Estas estrategias, inspiradas en el diseño de videojuegos, permiten transformar la experiencia educativa en una dinámica más interactiva y centrada en el estudiante. A continuación, se describen algunos de los elementos que más se utilizan para el LESL (Learning English as a Second Language) según una revisión sistemática de elementos de gamificación \cite{Dehghanzadeh2021}, la cual cuenta con el análisis de 96 artículos los cuales fueron filtrados de la siguiente manera para obtener 22  artículos que utilizan técnicas de gamificación para el LESL:

\newpage
\begin{itemize}
  \item Publicaciones que solo utilizan la gamificación era para LESL.
  \item Solo estudios en entornos digitales.
  \item Exclusión de investigaciones centradas en otros idiomas distintos al inglés.
\end{itemize}

\begin{table}[H]
\centering
\begin{tabular}{|l|p{10cm}|}
\hline
\textbf{Elemento de gamificación} & \textbf{Descripci\'on} \\ \hline
Retroalimentaci\'on (Feedback) & Informaci\'on inmediata sobre el desempe\~no del usuario. \\ \hline
Desaf\'io (Challenge) & Retos que ponen a prueba las habilidades del usuario. \\ \hline
Recompensa (Reward) & Beneficios otorgados por completar actividades. \\ \hline
Tabla de clasificaci\'on (Leaderboard) & Lista que muestra el rendimiento comparativo entre los usuarios. \\ \hline
Presi\'on de tiempo (Time pressure) & L\'imite temporal para completar una tarea. \\ \hline
Barra de progreso (Progress bar) & Indicador visual del avance en el aprendizaje. \\ \hline
Puntos (Point) & Unidad num\'erica que cuantifica logros. \\ \hline
Nivel (Level) & Etapa alcanzada seg\'un la acumulaci\'on de logros. \\ \hline
Insignia (Badge) & Distintivo visual que simboliza un logro. \\ \hline
Sistema de puntuaci\'on (Score system) & M\'etodo para calcular el desempe\~no total del estudiante. \\ \hline
Avatar & Representaci\'on gr\'afica del estudiante en el entorno digital. \\ \hline
Historia (Story) & Contexto narrativo que enmarca las actividades. \\ \hline
Narraci\'on (Narration) & Descripci\'on guiada de eventos dentro de la experiencia. \\ \hline
\end{tabular}
\caption{Técnicas comunes de gamificación y sus descripciones breves}
Fuente: Using gamification
to support LESL \cite{Dehghanzadeh2021}.
\end{table}


\subsection{Elementos de gamificación seleccionados}

Para el desarrollo del presente proyecto, el cual esta orientado al aprendizaje del inglés como lengua extranjera, se seleccionaron una combinación de técnicas gamificadas que se ajustan de manera adecuada al objetivo y funciones del prototipo.

\begin{itemize}
\item \textbf{Puntos de experiencia (XP): } como método para cuantificar el progreso tras cada lección completada.
\item \textbf{Insignias: } en función de la acumulación de XP, actuando como incentivo visual y simbólico.
\item \textbf{Días de racha: } para premiar la participación continua y fomentar la constancia en el aprendizaje.
\item \textbf{Ranking de jugadores: } que permite a los usuarios comparar su rendimiento y sentirse parte activa de una comunidad de aprendizaje.
\end{itemize}


