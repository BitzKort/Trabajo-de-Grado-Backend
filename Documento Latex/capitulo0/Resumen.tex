\begin{center}
	\section*{Resumen}	
\end{center}

El presente proyecto desarrolló un prototipo web diseñado para mejorar la habilidad del habla en inglés mediante el uso de técnicas de Procesamiento del Lenguaje Natural (PLN) y gamificación. Se aborda la creciente necesidad de herramientas educativas efectivas que involucren a los estudiantes de manera activa y atractiva. El sistema proporciona ejercicios interactivos y elementos de gamificación para motivar a los usuarios a practicar y mejorar su comprensión al momento de hablar en inglés. El enfoque que tiene el sistema aporta al estudiante una herramienta con IA centralizada en distintos algoritmos de PLN con el fin de que el estudiante obtenga un apoyo en el proceso de aprendizaje y que, con ayuda de estos modelos matemáticos, pueda mejorar su fluidez al momento de comunicarse en inglés.
\\
\\
 \textbf{Palabras claves:}
Procesamiento del Lenguaje Natural, Educación, Juegos, Ciencias de la Computación, IA

\newpage
\begin{center}
	\section*{Abstract}
\end{center}

The present project developed a web prototype designed to improve English speaking skills through the use of Natural Language Processing (NLP) techniques and gamification. It addresses the growing need for effective educational tools that engage learners in an active and engaging way. The system provides interactive exercises and gamification elements to motivate users to practice and improve their understanding when speaking English. The system's approach provides the student with an AI tool centered on different PLN algorithms in order for the student to obtain support in the learning process and, with the help of these mathematical models, to improve their fluency when communicating in English.
\\
\\
 \textbf{Keywords:}
Natural Language Processing, Education, Games, Computer Science, AI
 

 