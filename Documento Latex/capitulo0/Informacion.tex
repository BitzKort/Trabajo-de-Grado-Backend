\begin{center}
	\section*{Información sobre los capítulos
}	
\end{center}

En la siguiente tabla se muestra una corta descripción de cada capítulo del documento. 

\begin{longtable}{|p{4.5cm}|p{8.5cm}|p{2.5cm}|}
\cline{1-3} 
\cellcolor[gray]{0.9} \textbf{Capítulo} & \cellcolor[gray]{0.9}\textbf{Descripción}\\
\cline{1-3}
Capítulo 1: {Introducción} & {En este capítulo se presenta el contexto general del proyecto, se establecen los objetivos generales y específicos que guiarán el desarrollo del trabajo.}\\

\cline{1-3}
Capítulo 2: {Alcance} & {En este capítulo se define el alcance del proyecto.}\\

\cline{1-3}
Capítulo 3: {Marco Referencial} & {Se explican las teorias y  conceptos necesarios para comprender el problema a resolver del proyecto y su solución.}\\

\cline{1-3}
Capítulo 4: {Análisis de Fundamentos} & {Se desarrolla el objetivo especifico 1 el cual interioriza la selección de las técnicas de gamificación, algoritmos de procesamiento del lenguaje natural, metodologías de enseñanza y temas de ingles escogidos para el prototipo}\\

\cline{1-3}
Capítulo 5: {Diseño de artefactos} & {Se desarrolla el objetivo especifico 2 el cual se especializa en el diseño de los artefactos del prototipo, como los mockups de la interfaz de usuario, el esquema de la base de datos y el diagrama de la arquitectura web del prototipo }\\

\cline{1-3}
Capítulo 6: {Desarrollo del prototipo web} & {Se desarrolla el objetivo especifico 3 el cual se especializa en la implementación de los resultados de los objetivos anteriores en el prototipo web}\\

\cline{1-3}
Capítulo 7: {Pruebas sobre el prototipo} & {Se desarrolla el objetivo especifico 4, donde se presentan las pruebas de uso que se realizaron con la población objetivo.}\\

\cline{1-3}

Capítulo 8: {Conclusiones y trabajos futuros} & {Las conclusiones finales del proyecto y los trabajos futuros que deben realizarse para mejorar el prototipo web en éste trabajo de grado.}\\
\cline{1-3}
\caption{Estructura de capítulos}\\
\end{longtable}

\begin{center}
\end{center}
