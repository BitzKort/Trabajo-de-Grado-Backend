\section{Declaración del alcance}

El presente proyecto desarrolló de un prototipo web con la finalidad de brindar ayuda al aprendizaje de inglés utilizando PLN y Gamificación para personas que deseen mejorar sus conocimientos de este idioma en Colombia con un nivel de conocimiento base de A2, dentro de un plazo de 8 meses, utilizando un temario según los niveles de inglés descritos según el Marco Común Europeo de Referencia para las lenguas (MCER), basado en un nivel B1 \cite{mcer2002} como guía para su enseñanza. Se implementarán técnicas de procesamiento de lenguaje natural para el desarrollo y funcionamiento del sistema, además contará con mecánicas de gamificación para que los usuarios se apropien del conocimiento de una manera de reto y desarrollen su competitividad con el fin de adquirir y ampliar el uso cotidiano del inglés.
\\
\\
Es necesario recalcar que para el idóneo desarrollo del proyecto se tienen ciertas restricciones y especificaciones que se deben de plantear y tener en cuenta, los cuales son:

\subsection{Supuestos}

\begin{itemize}
    \item Normalidad académica durante el desarrollo del proyecto.
    \item Continuidad del director del trabajo de grado.
    \item No ocurrencia de eventos imprevistos durante la realización del proyecto.
    \item Acceso a la información necesaria para el proyecto, así como la implementación de las tecnologías.
\end{itemize}
\subsection{Restricciones}

\begin{itemize}
    \item El tiempo de desarrollo del proyecto es de ocho (8) meses, los cuales comprenden un total de dos (2) períodos académicos.
    \item El hardware utilizado para el proyecto será el que esté a disposición del estudiante.
    \item Se implementarán un mínimo de tres (3) algoritmos de procesamiento de lenguaje natural.
    \item Se implementarán un mínimo de dos (2) técnicas de gamificación para cumplir el fin del proyecto.
\end{itemize}
\vfill
