\newpage
\section{Justificación}

El mejoramiento de la comprensión oral en ingles de la población bilingüe en Colombia es un desafió de gran relevancia tanto nivel académico como socioeconómico. El poco dominio de la lengua inglesa contribuye a la perdida de oportunidades laborales y académicas, tal como lo evidencian los puntajes promedio del examen saber 11 (48 para el calendario A y 72 para el calendario B en 2020) pues este examen sirve como prueba de que los jóvenes pre-universitarios no cuentan con un nivel satisfactorio para pasar exámenes de admisión de universidades que estén en este idioma O que tengan una prueba del dominio del mismo, esto trae consigo que el estudiante pierda, en primera instancia, las nuevas oportunidades académicas que se le presentan. Elevar estos resultados aumentaría y posicionaría a los jóvenes colombianos en mejores condiciones competitivas dentro de un mercado laboral globalizado.
\\
\\
Desde el ámbito educativo, es fundamental promover entornos prácticos de inmersión más allá del aula, que permitan incorporar el uso del inglés en la vida diaria y establecer una rutina constante de práctica del idioma, evitando metodologías monótonas y/o ortodoxas que no promueven la asimilación natural de la lengua. Al implementar soluciones tecnológicas activas que faciliten la practica continua (plataformas de interacción oral, videojuegos o comunidades de intercambio cultural) permitirá mantener la motivación y reforzar la apropiación del idioma, evitando el desuso del conocimiento adquirido de este idioma.
\\
\\
El fortalecimiento del ingles contribuye directamente al desarrollo social y económico del país, Las empresas adoptan cada vez mas políticas bilingües con el objetivo de obtener un mayor alcance, ya que el uso del ingles en el ámbito laboral trae consigo negociaciones internacionales y proyectos de investigación en conjunto con otras naciones, lo que resulta en una mayor competitividad en el mercado \cite{cronquist2017aprendizaje}. Un mayor numero de profesionales con competencia oral avanzada impulsaría así la llegada de inversión extranjera, el intercambio cultural y una participación mas activa de Colombia en un mercado global y académico.
\\
\\
Aprovechar las infraestructuras digitales ya existentes (internet, dispositivos móviles, computadores portátiles, tabletas, etc) responde a la necesidad de un aprendizaje flexible (bleend learning).
\\
\\
El desarrollo de este proyecto busca ofrecer una solución innovadora que facilite la practica constante del idioma ingles a través de sistemas que implementen modelos e procesamiento de lenguaje natural. Al implementar una plataforma que estimule la fluidez oral del idioma, se genera oportunidades reales de inmersión lingüística para las personas carentes de espacios adecuados para afianzar sus conocimientos. Esto no solo fortalecerá las competencias comunicativas en ingles de la población bilingüe colombiana, sino que también contribuirá a cerrar brechas académicas, sociales y culturales.
