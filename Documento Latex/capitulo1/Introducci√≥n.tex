El dominio de habilidades lingüísticas, especialmente en idiomas ampliamente utilizados como el inglés, se ha vuelto crucial tanto a nivel personal como profesional. Siendo el inglés uno de los lenguajes más utilizados en el mundo, se torna en muchas ocasiones como un punto decisivo en el ámbito profesional \cite{eaii}. 
\\
\\
El aprendizaje de la gramática ha cambiado con el paso de los tiempos y, junto a los avances tecnológicos que se han presentado en las últimas décadas, han permitido una mayor integración de la tecnología en el proceso educativo. En lugar de depender exclusivamente de los métodos de enseñanza tradicionales, como lo son los libros de texto y ejercicios escritos, los estudiantes ahora tienen acceso a una amplia gama de recursos tecnológicos que ofrecen experiencias de aprendizaje más interactivas y personalizadas, convirtiéndose en un aprendizaje combinado (Blended Learning) \cite{salinas2010entrenamiento}, destacando los programas de software y las aplicaciones móviles diseñadas específicamente para la enseñanza del idioma inglés.
\\
\\
Entre los avances tecnológicos más destacados en el aprendizaje de la gramática en inglés se encuentra el uso de técnicas de procesamiento del lenguaje natural (PLN). Permite a los programas y aplicaciones analizar el texto de manera que, identificando patrones gramaticales, partes del discurso (sustantivos, verbos, pronombres, etc) y proporcionando retroalimentación instantánea sobre la gramática \cite{alhawiti2014natural}, los estudiantes complementan su aprendizaje y este sea mucho mas rapido y facil para ellos. Sin embargo, al momento en que el estudiante se siente conforme con lo aprendido, muchas veces no obtiene una recompensa, o gratificación alguna, lo cual es una situacion muy comun al momento de pasar por el proceso de aprendizaje, ya que el modelo de enseñanza tradicional no cuenta con algún tipo de recompensas, más allá de dar una nota que, en algunos casos podría ser todo lo contrario. La gamificación es la utilización de elementos de diseño de videojuegos en contextos que no son de juego, por lo cual al utilizar gamificación en el transcurso de aprendizaje de un idioma, aprovechamos el Blended Learning como apoyo para añadir estrategias motivacionales tales como puntos, rachas y logros, los cuales toman como referencia elementos de los videojuegos \cite{flores2015using}.
\\
\\
Una de las principales dificultades que enfrenta latinoamérica en el aprendizaje del inglés radica en su escasa aplicación en la vida cotidiana, la falta de exposición constante al idioma dificulta el desarrollo de habilidades comunicativas, no es de extrañar que el dominio del inglés en américa latina está por debajo de los estándares internacionales \cite{cronquist2017aprendizaje}.
\\
\\


 