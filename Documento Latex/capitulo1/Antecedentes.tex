\newpage
\section{Antecedentes}

\textbf{Elsa Speak: }
La inteligencia artificial se ha convertido en un tema a tratar en la enseñanza de  lenguajes por la razón de que puede asistir y mejorar el aprendizaje de lenguajes para todos los niveles de educación. ELSA Speak es un aplicación de Automatic Speech Recognition (ASR) usada para enseñar pronunciación, estudia cómo los estudiantes escuchan, hablan, vocalizan y aciertan palabras en inglés basado un lenguaje oral. ELSA Speak puede incrementar las habilidades en el idioma inglés de los estudiantes, llegando a motivarlos a continuar su proceso de formación y/o aprendizaje \cite{kholis2021elsa}.
\\
\\
\textbf{Duolingo: }
El mobile learning es un tipo de actividad de aprendizaje que se sirve a través de dispositivos móviles los cuales no requieren que el estudiante esté en una ubicación geográfica en particular, Duolingo es una de las aplicaciones MALL (Mobile-Assisted language learning) más dominantes e influyentes en este campo, siendo una fuerte representación del uso de gamificación para la enseñanza de idiomas, implementando un sistema de retos, recompensas,  un sistema de niveles y de clasificación basados en los logros que obtiene la el estudiante a medida que va aprendiendo un determinado idioma \cite{shortt2023gamification}.
\\
\\
\textbf{ChatGPT: }
El rápido avance de la inteligencia artificial ha brindado soporte a diversos campos incluido el aprendizaje de idiomas, los sistemas de chat basados en IA, como ChatGpt han demostrado su potencial en el desarrollo de habilidades comunicativas en el idioma inglés. Este chat da la oportunidad a los estudiantes de tener interacciones de lenguaje natural en un entorno simulado, esto se da gracias a la implementación de algoritmos de PLN y ML. El uso de este sistema puede mejorar la confianza y la competencia comunicativa de los estudiantes interesados \cite{chicaiza2023aplicaciones}.
\\
\\
\textbf{VIDEOJUEGO PARA EL APOYO A LA ENSEÑANZA DE INGLÉS PARA LOS ESTUDIANTES DE INGENIERÍA DE SISTEMAS DE LA UNIVERSIDAD DEL VALLE SEDE TULUÁ:  }
Trabajo de grado del Ingeniero de Sistemas Luiz Fernando Quintero Castaño, en el cual aborda el apoyo a la enseñanza de inglés en base a un videojuego, utilizando técnicas de gamificación para obtener la “experiencia de juego”, Assessing the core Elements of the Gaming Experience \cite{calvillo2015assessing} juega un papel importante en las técnicas de gamification utilizadas en el proyecto \cite{quintero2018videojuego}.
\\
\\
\textbf{Prototipo de una aplicación de procesamiento de lenguaje natural aplicando traducción a lengua de señas colombiana en dispositivos móviles: }
Trabajo de grado de los Ingenieros de Sistemas Alejando Azcarate Trujillo y Karem Lizeth Torres Ríos ambos Egresados de la Universidad del Valle sede Tuluá, en el cual aborda su proyecto en base al diseño, modelado e implementación de un prototipo de aplicación móvil utilizando llamado Speech To Sings el cual utiliza Modelos Ocultos de Markov  con la finalidad de apoyar el proceso de comunicación entre las personas con discapacidad auditiva \cite{azcarate2020prototipo}.
\\
\\
\textbf{Objetos Virtuales de Aprendizaje con elementos de gamificación para el Apoyo de la Asignatura Fundamentos de Programación del Programa Académico de Ingeniería de Sistemas de la Universidad del Valle Sede Tuluá: }
Trabajo de grado de los Ingenieros de Sistemas Paula Andrea Corre Villalobos y Marisol Davila Escobar ambos Egresado de la Universidad del Valle sede Tuluá, en el cual se busca utilizar Objetos Virtuales de Aprendizaje (OVAs) [16] que contribuyan en la disminución de la deserción de la Asignatura académica Fundamentos de Programación de la Universidad del Valle sede Tuluá. Las técnicas de gamificación utilizadas en el proyecto se basan en la pirámide de los Elementos de Gamificación de Kevin Werbach \cite{correa2018objetos}.